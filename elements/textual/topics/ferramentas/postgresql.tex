\section{PostgreSQL}
\label{sec:postgresql}
PostgreSQL trata-se de um sistema gerenciador de banco de dados de código aberto, muito famoso e utilizado amplamente no âmbito empresarial. Esse sistema caracteriza-se principalmente por sua arquitetura robusta, fácil instalação, uso prático e possibilidade de utilização de extensões \cite{PostgreSQL2023}. Além disso, esse sistema incrementa a linguagem padrão de manipulação de banco de dados relacionais (SQL), oferecendo ferramentas de escala e reserva de cargas de trabalho de dados, sendo utilizado principalmente para coordenar dados de aplicativos \textit{web}, móveis, geospaciais e analíticos \cite{Kinsta2023}.

\begin{citacao}
    "O PostgreSQL conquistou uma forte reputação por sua arquitetura comprovada, confiabilidade, integridade de dados, conjunto robusto de recursos, extensibilidade e a dedicação da comunidade de código aberto por trás do software para fornecer consistentemente soluções inovadoras e de alto desempenho. O PostgreSQL é executado em todos os principais sistemas operacionais , é compatível com ACID desde 2001 e possui complementos poderosos, como o popular extensor de banco de dados geoespacial PostGIS."

    \cite{PostgreSQL2023}.
\end{citacao}

Normalmente, a maioria dos banco de dados relacionais podem ser melhor descritos como sistemas de gerenciamento de banco de dados relacionais (SGBDR), projetados para armazenar os dados em estruturas semelhantes a tabelas com colunas e tipos de dados pré-definidos, podendo serem consultados, modificados e recuperados usando técnicas baseadas em álgebra relacional. Entretanto, PostgreSQL tecnicamente é um sistema de gerenciamento de banco de dados objeto-relacionais (SGBDOR). O que na prática, além de possuir as características que os outros sistemas possuem, PostgreSQL também permite \cite{Prisma2020}:

\begin{itemize}
    \item Definir tipos de dados complexos customizados;
    \item Delimitar relacionamento de herança entre as tabelas;
    \item Criar funções de sobrecarga para trabalhar com diferentes tipos de argumento.
\end{itemize} 

Por tratar-se de um \textit{software} de código aberto, PostgreSQL, ao contrário de outros produtos similares, é desenvolvido e gerenciado exclusivamente pelo \textit{The PostgreSQL Global Development Group}, não possuindo um proprietário corporativo ou contraparte comercial. Isso permite que os colaboradores do grupo possam traçar e trabalhar em recursos com os quais a comunidade mais se preocupa. Os serviços profissionais para a ferramenta são providas por empresas externas que frequentemente contribuem para o projeto, mas que não controlam o processo de desenvolvimento \cite{Prisma2020}.

Em respeito a sua arquitetura, PostgreSQL estrutura-se de maneira simplória e efetiva, possuindo uma memória compartilhada, processos de \textit{background} e sistema de diretório de dados. Em um fluxo usual, o cliente realiza uma solicitação ao servidor, que então processa os dados utilizando buffers compartilhados e processos em segundo plano \cite{Kinsta2023}.
