\chapter{INTRODUÇÃO}
\label{chap:introducao}

Desde o final do ano de 2019, com a pandemia mundial do vírus SARS-CoV-2, a mudança repentina e praticamente permanente da rotina e das relações sociais trouxe notória intensificação de assuntos pautados e direcionados à saúde mental, assim como os subsequentes impactos no mercado de psicologia.

Em primeiro plano, ao analisar o interesse público em saúde mental, constata-se que o mesmo obteve palpável crescimento. Este crescimento fundamenta-se na pesquisa \textit{World Mental Health Day}, realizada pelo instituto \citeonline{IPSOS2021}, que evidencia a saúde mental como o terceiro problema mais significativo de saúde pública de acordo com a população dos 30 países envolvidos, ficando atrás apenas do COVID-19 e do câncer respectivamente. Além disso, no ano seguinte, a pesquisa \textit{Global Health Service Monitor}, realizada pelo mesmo instituto, constata que o Brasil é o sétimo país que mais relata preocupação com saúde mental, onde 49\% dos brasileiros entrevistados a consideram como o problema de saúde mais enfrentado no país.

Nessa mesma circunstância, o aumento do interesse público no assunto também reflete em crescimento do mercado de psicologia. Em âmbito nacional, a \citeonline{ABP2020}, realizou uma pesquisa que constatou que 47\% dos profissionais entrevistados relataram aumento em seus atendimentos após o início da pandemia, e deste grupo, para cerca de um terço, os atendimentos cresceram em até 25\%. Já em contexto estrangeiro, a pesquisa \textit{COVID-19 Practitioner Impact Survey}, realizada pela \citeonline{Association2022}, ratifica o aumento do mercado, evidenciando que 43\% dos entrevistados constataram estar trabalhando mais em relação ao início da pandemia, além de também expor que 4 em cada 10 profissionais relatam manter uma lista de espera de atendimento de tamanhos variados.

Além disso, em decorrência dos impactos causados pela pandemia, nota-se que este mercado possui uma aguda tendência de integração com a tecnologia. De acordo com o resumo científico \textit{Mental Health and COVID-19: Early evidence of the pandemic’s impact}, publicado pela \citeonline{OMS2022}, a disrupção causada principalmente pela pandemia SARS-CoV-2 levou a profissionais da área de saúde mental relatarem cortes abruptos por inúmeros motivos em acompanhamentos psicológicos de seus pacientes, e que para mitigar os danos, grande parte dos profissionais da área alegaram que passaram a integrar em seus serviços inúmeras tecnologias digitais. Aliado a isso, inúmeros estudos avaliam a eficácia do emprego da terapia online é muito similar  é muito similar ao que é alcançado pelos métodos tradicionais, sendo este o mais recomendado quando a abordagem presencial não se faz presente por quaisquer motivos.

Pelos fatores citados, o presente trabalho propõe-se a desenvolver um sistema \textit{web} chamado PsyCare, que colabore e facilite com a maior parte do processo de acompanhamento psicológico. Para isso, em primeiro plano, para facilitar a busca inicial do paciente, a plataforma possibilita a listagem de profissionais cadastrados e validados, como também realizar agendamento de sessões de acordo com a agenda livre do psicólogo e paciente. Além disso, a fim de facilitar e centralizar a comunicação entre ambas as partes, a plataforma também possui consultas por videoconferência e chat entre as partes. Por fim, o sistema também integra prontuário médico, lista de afazeres e histórico das sessões para as partes envolvidas.

Primeiramente, no capítulo 2, elucida-se a conjuntura atual da área de psicologia, explicando as principais abordagens psicológicas utilizadas pelos profissionais da área, seguida do estudo dos impactos causados na área pela pandemia.

Logo em seguida, no capítulo 3, apresenta-se as ferramentas e tecnologias utilizadas para o desenvolvimento deste trabalho, destacando-se o NodeJS, ReactJS, NestJS e PostgreSQL.

Após isso, no capítulo 4, descreve-se um pequeno estudo sobre as ferramentas já existentes no mercado que possuem objetivo e/ou funcionalidades similares ao PsyCare, seguido do detalhamento das funcionalidades do sistema através de diagramas UMLs, e por fim, uma breve descrição do desenvolvimento do trabalho.

O capítulo 5 contém a descrição das telas e funcionalidades especificadas e desenvolvidas no trabalho, alinhado ao conjunto de imagens respectivas do sistema; E por fim, no sexto e último capítulo, é apresentada a conclusão final, os resultados obtidos e os possíveis trabalhos futuros.
