\chapter{CONCLUSÃO E TRABALHOS FUTUROS}
\label{chap:conclusao}

Por meio do desenvolvimento deste trabalho, principalmente durante a etapa de levantamento de requisitos, tornou-se possível compreender mais profundamente todo o processo de acompanhamento psicológico, assim como também aprender sobre as principais abordagens psicológicas utilizadas atualmente. Nesse contexto, em decorrência dos impactos sociais e psicológicos causados pela pandemia do COVID-19, o mercado de psicologia passou por profunda transformação e desenvolvimento, integrando-se cada vez mais ao meio tecnológico para conseguir comportar todas as mudanças.

Dentro desse cenário, este trabalho procura evoluir concomitantemente com as mudanças da área e com todos seus avanços tecnológicos. Além dos conhecimentos de psicologia obtidos, durante o período de desenvolvimento da aplicação foi possível compreender mais profundamente inúmeras tecnologias utilizadas ativamente no mercado de trabalho, que ao serem aplicadas juntamente com os conhecimentos obtidos durante o curso, principalmente aos ligados às disciplinas de Redes, Banco de Dados, Engenharia de Software e Lógica de Programação, foi possível obter uma visão mais abrangente e real de como o desenvolvimento de software funciona na prática.

Além disso, durante o desenvolvimento da plataforma, alguns pontos trouxeram maior dificuldade a serem implementados, exigindo assim maior atenção e tempo. Como por exemplo, a implementação da videochamada, que mesmo estando abstraída através das tecnologias escolhidas, ainda necessitou de um profundo conhecimento do protocolo \textit{WebRTC},  mas com a extensa documentação presente na internet foi-se possível sanar toda a problemática eficientemente.

Ao comparar os objetivos propostos durante o período de análise de requisitos e de projeto do trabalho, percebe-se os mesmos foram cumpridos, uma vez que as características desenvolvidas e entregues pelo sistema auxiliam e facilitam a maior parte do processo de acompanhamento psicológico, fornecendo um sistema amparador para o paciente ao mesmo tempo que proporciona uma plataforma centralizadora e facilitadora ao profissional.

Em respeito a possíveis trabalhos futuros, o plano inicial consiste em melhorar funcionalidades já existentes na ferramenta, focando-se em dois fluxos: melhorar a experiência de usuário no gerenciamento das sessões, tornado-as mais flexíveis e rever a possibilidade de uma mesma pessoa ter a capacidade de representar ambos os papéis da aplicação, sendo assim um “Usuário” e “Profissional” ao mesmo tempo. Já em segundo plano, pretende-se adicionar novas funcionalidades à ferramenta, onde a principal a ser adicionada é um sistema de notificações, onde tanto profissional quanto usuário receberá notificações de acontecimentos relevantes (como o aceite/encerramento de um atendimento, lembretes de encontros, novas mensagens recebidas, dentre outros). Por fim, em um terceiro momento, planeja-se transformar a aplicação em um sistema multiplataforma, passando a dar suporte também à sistemas \textit{mobile}.
