\section{Implementação}
\label{sec:implementacao}

Esse subcapítulo tem como objetivo descrever brevemente a implementação prática deste trabalho, a escrita será dividida entre \textit{back-end} com NestJS e \textit{front-end} com \textit{ReactJS}. Como requisitos básicos, há o \textit{NodeJS}, \textit{PostgreSQL}, uma IDE (como o \textit{Visual Studio Code}) e uma \textit{API Platform} (como por exemplo, \textit{Postman}) instalados. 

\subsection{API back-end}
\label{sec:apiBackEnd}
Para iniciar o desenvolvimento da API com \textit{NestJS} primeiramente é necessário instalar a biblioteca CLI. Com a CLI instalada, inicia-se o esqueleto da aplicação, para logo após instalar as bibliotecas auxiliares para dar suporte ao Postgresql, juntamente com o mapeamento objeto-relacional e suas devidas validações. Todos os comandos são executados em um terminal, da seguinte maneira:
\begin{lstlisting}[language=bash]
    npm install -g @nestjs/cli
    nest new nomeDoProjeto
    cd nomeDoProjeto/
    npm install --save @nestjs/typeorm typeorm pg class-validator class-transformer
\end{lstlisting}

Partindo disso, inicia-se a codificação utilizando qualquer IDE de desenvolvimento. O primeiro passo é adicionar a importação e configuração do TypeORM (biblioteca responsável por conectar a API ao banco de dados) no arquivo \textSpecial{app.module.ts}, da seguinte forma:
\begin{lstlisting}[language=bash]
import { Module } from '@nestjs/common';
import { AppController } from './app.controller';
import { AppService } from './app.service';
import { TypeOrmModule } from '@nestjs/typeorm';

@Module({
    imports: [
    TypeOrmModule.forRoot({
        type: 'postgres',
        host: 'Host',
        port: Porta,
        username: 'Usuario',
        password: 'Senha',
        database: 'Nome do banco de dados',
        synchronize: true,
        logging: false,
        autoLoadEntities: true,
    }),
    ],
    controllers: [AppController],
    providers: [AppService],
})
export class AppModule {}    
\end{lstlisting}

Ao ter a instalação e configuração concluídas, parte-se diretamente para o gerenciamento dos recursos da aplicação. Dentro desse contexto, considera-se que cada tabela do banco de dados é um recurso isolado. Para criar um novo recurso utiliza-se a CLI instalada anteriormente, executando \textSpecial{nest generate resource user}, selecionando as opções “REST API” e “Yes”, respectivamente.

Com o recurso criado, altera-se o arquivo \textSpecial{user.entity.ts}, sobre o qual é responsável por representar a entidade e tabela user. Portanto, adiciona-se as colunas necessárias, conforme o exemplo abaixo:
\begin{lstlisting}
import { Column, Entity, PrimaryGeneratedColumn } from 'typeorm';

@Entity()
export class User {
    @PrimaryGeneratedColumn()
    id: number;

    @Column({ type: 'varchar' })
    name: string;

    @Column({ type: 'varchar' })
    surname: string;

    @Column({ type: 'varchar' })
    email: string;

    @Column({ type: 'varchar' })
    password: string;

    @Column({ type: 'int', nullable: true })
    age: number;

    constructor(name: string, surname: string, email: string, password: string, age?: number) {
        this.name = name;
        this.surname = surname;
        this.email = email;
        this.password = password;
        this.age = age;
    }
}
\end{lstlisting}

Além disso, altera-se o arquivo \textSpecial{user.module.ts} para importar a nova entidade para dentro da aplicação, e consequentemente para dentro do banco de dados. O processo é dado da seguinte forma:
\begin{lstlisting}
import { Module } from '@nestjs/common';
import { UserService } from './user.service';
import { UserController } from './user.controller';
import { TypeOrmModule } from '@nestjs/typeorm';
import { User } from './entities/user.entity';

@Module({
 imports: [TypeOrmModule.forFeature([User])],
 controllers: [UserController],
 providers: [UserService],
})
export class UserModule {}
\end{lstlisting}

Após o processo descrito, desenvolve-se o objeto de transferência de dados (DTO), que, de maneira simplista, descreve a forma com que os dados devem ser informados ao servidor. O arquivo \textSpecial{create-user.dto.ts} fica assim:
\begin{lstlisting}
import { IsEmail, IsInt, IsNotEmpty, IsString } from 'class-validator';

export class CreateUserDto {
    @IsString()
    @IsNotEmpty()
    name: string;

    @IsNotEmpty()
    @IsNotEmpty()
    surname: string;

    @IsNotEmpty()
    @IsEmail()
    email: string;

    @IsInt()
    age: number;

    @IsString()
    @IsNotEmpty()
    password: string;
}
\end{lstlisting}

O próximo passo é desenvolver o \textSpecial{user.service.ts}, que conterá as lógicas de execução dos CRUDs, onde no nosso contexto, será apenas a criação e leitura:
\begin{lstlisting}
import { Injectable } from '@nestjs/common';
import { CreateUserDto } from './dto/create-user.dto';
import { User } from './entities/user.entity';
import { InjectRepository } from '@nestjs/typeorm';
import { Repository } from 'typeorm';

@Injectable()
export class UserService {
    constructor(@InjectRepository(User) repository: Repository<User>) {}

    create(dto: CreateUserDto) {
        const user: User = new User(
            dto.name,
            dto.surname,
            dto.email,
            dto.password,
            dto.age
        );

        return this.repository.save(user);
    }

    findOne(id: number) {
        return this.repository.findOne({ where: { id } });
    }
}
\end{lstlisting}

Por último, desenvolve-se as rotas que irão consumir as lógicas criadas no passo anterior, as alterações serão no arquivo \textSpecial{user.controller.ts}:
\begin{lstlisting}
import { Controller, Get, Post, Body, Param } from '@nestjs/common';
import { UserService } from './user.service';
import { CreateUserDto } from './dto/create-user.dto';

@Controller('user')
export class UserController {
    constructor(private readonly userService: UserService) {}

    @Post()
    create(@Body() createUserDto: CreateUserDto) {
        return this.userService.create(createUserDto);
    }

    @Get(':id')
    findOne(@Param('id') id: string) {
        return this.userService.findOne(+id);
    }
}
\end{lstlisting}

Com tudo feito até aqui, executa-se a API com o comando \textSpecial{npm run start:dev}. Para testar as rotas recém criadas na \textit{API Plataform}. A \refImage{backEnd} demonstra esse fluxo.
\image
    {Consumo da API de exemplo}
    {backEnd}
    {data/figures/backend.png}
    {width=1\textwidth}
    {Autor}

\subsection{Interface front-end}
\label{sec:frontEnd}
Para começar o desenvolvimento da interface do usuário, basta rodar os comando de instalação e criação do esqueleto da aplicação em ReactJS em qualquer terminal:
\begin{lstlisting}[language=bash]
npx create-react-app nomeDoProjeto --template typescript
cd nomeDoProjeto/
npm install @mui/material @emotion/react @emotion/styled @mui/icons-material    
\end{lstlisting}

Com a instalação concluída, torna-se possível começar a desenvolver a tela no arquivo \textSpecial{App.tsx}, conforme mostra o seguinte código:
\begin{lstlisting}
import './App.css';
import { AppBar, Button, IconButton, Toolbar, Typography } from '@mui/material';
import PsychologyIcon from '@mui/icons-material/Psychology';

function App() {
    return (
        <div className="app">
            <AppBar position="static">
            <Toolbar>
                <IconButton sx={{ mr: '1rem', color: 'white' }}>
                    <PsychologyIcon fontSize='large' />
                </IconButton>
                <Typography variant="h6" component="div" sx={{ flexGrow: 1 }}>
                    Exemplo PsyCare
                </Typography>
                <Button color="inherit">Login</Button>
            </Toolbar>
            </AppBar>
            <Typography variant='h5' sx={{ marginTop: '1rem', marginLeft: '2rem' }}>
                Seja bem-vindo(a) ao PsyCare!
            </Typography>
        </div>
    );
}
export default App;
\end{lstlisting}

Com o desenvolvimento concluído, executa-se a aplicação através do comando npm start, que ao ser executado, abrirá uma nova janela no navegador padrão do sistema, com a aplicação aberta. A \refImage{frontEnd} demonstra a interface de demonstração que foi montada.
\image
    {Interface de demonstração}
    {frontEnd}
    {data/figures/frontend.png}
    {width=1\textwidth}
    {Autor}