\section{Tela de chamada}
\label{sec:chamada}

Estando autenticado como usuário ou como profissional, a fim de permitir melhor comunicação entre paciente e psicólogo, a tela de chamada permite aos integrantes do acompanhamento realizar videochamadas. Antes de participar propriamente da chamada, a aplicação apresenta uma tela de confirmação, orientando o profissional/usuário em estar em um ambiente minimamente tranquilo e com privacidade, visto que tais características são essenciais para a eficácia e confiança da sessão terapêutica. A \refImage{callConfirmation} demonstra esse aviso.

\image
    {PsyCare - Chamada, confirmação}
    {callConfirmation}
    {data/figures/call-confirmation.png}
    {width=1\textwidth}
    {Autor}

Ao clicar em \quotes{Entrar}, o paciente/profissional passa a participar da videochamada. Aqui, a aplicação disponibiliza as funções que uma videochamada normalmente possui, como habilitar/desabilitar o microfone e câmera (podendo escolher qual dispositivo utiliza-se em cada um deles), assim como também disponibiliza o um bate papo integrado. Partindo disso, a \refImage{call} exemplifica essa tela.

\image
    {PsyCare - Chamada}
    {call}
    {data/figures/call.png}
    {width=1\textwidth}
    {Autor}

As videochamadas buscam dar apoio aos encontros semanais que os integrantes possuem de seus atendimentos, entretanto, para facilitar ainda mais a comunicação, a plataforma permite chegar à tela de chamada em qualquer momento do dia. 