\begin{resumo}[RESUMO]
\begin{SingleSpacing}

O presente trabalho, denominado PsyCare, propõe-se a desenvolver uma plataforma web de acompanhamento psicológico, auxiliando tanto o paciente quanto o profissional durante a maior parte do processo da terapia, uma vez que a pandemia SARS-CoV-2 trouxe mudanças repentinas e praticamente permanentes na rotina e nas relações sociais, resultando em notória intensificação de interesse público em assuntos pautados e direcionados à saúde mental. Esse aumento reflete-se diretamente em um crescimento perceptível do mercado de psicologia, que devido aos impactos causados pela pandemia, também passou a apresentar uma aguda tendência de integrar-se tecnologicamente. Em relação a implementação, utilizou PostgreSQL no gerenciamento do banco de dados, integrado com NestJS para o desenvolvimento da API \textit{back-end} e ReactJS para implementar as interfaces \textit{front-end}. Ao final deste trabalho, foi possível apresentar um sistema acessível e eficaz que auxilia, organiza e centraliza o processo de acompanhamento terapêutico. \\

\textbf{Palavras-chave}: Atendimento psicológico. Sistema \textit{web}. NestJS. ReactJS.

\end{SingleSpacing}
\end{resumo}

