\section{Análise de mercado e proposta de novo sistema}
\label{sec:analise}

Partindo das conversas informais realizadas tanto com pacientes quanto com profissionais envolvidos no processo de acompanhamento psicológico, duas principais ferramentas foram citadas como plataformas já existentes com um propósito parecido com o do PsyCare. Ambas possuem características únicas e apresentam-se como ferramentas sólidas e muito bem empregadas dentro dessa área de mercado, entretanto, possuem certas limitações ou desvantagens. Com isso em mente, este subcapítulo propõe-se em detalhar brevemente essas duas plataformas, e logo após descrever a proposta de novo sistema.

Primeiramente, a plataforma Zenklub trata-se de uma plataforma de terapia online, destinada a ser utilizada por pacientes, profissionais da área e empresas. Dentre todas as funcionalidades presentes, as que mais se destacam são: facilidade em encontrar o especialista que mais se encaixa nas necessidades do usuário, assim como agendar facilmente uma sessão com o mesmo, gerenciamento de um diário emocional, e videochamadas integradas dentro da plataforma. Aliada a tudo isso, a interface simples e efetiva da ferramenta facilita imensamente a sua utilização, sendo extremamente convidativa ao usuário final. Além disso, a plataforma está disponível para a \textit{web} e sistemas \textit{mobile} (android e iOS). Na \refImage{zenklub} é possível visualizar parte da interface de pesquisa de profissionais.
\image
    {Interface da plataforma Zenklub}
    {zenklub}
    {data/figures/zenklub.png}
    {width=.9\textwidth}
    {Autor}

Além disso, a ferramenta Psitto apresenta-se como uma plataforma de terapia online, destinada para pacientes e profissionais da área. A ferramenta destaca-se por facilitar a busca de profissionais capacitados e validados pela plataforma e as videochamadas integradas dentro da plataforma. A plataforma está disponível para a \textit{web}. Na \refImage{psitto} é possível visualizar a página de pesquisa dos profissionais da plataforma.
\image
    {Interface da plataforma Psitto}
    {psitto}
    {data/figures/psitto.png}
    {width=.9\textwidth}
    {Autor}

Agregando as conversas informais com pacientes e profissionais juntamente com a análise das duas ferramentas citadas, identificou-se necessidades não supridas pelas ferramentas e pontos de possíveis melhorias. Assim, conclui-se que há necessidade do desenvolvimento de um novo sistema que, além de possuir as características mais eficazes de ambas as ferramentas (como o layout simplificado e efetivo e a busca facilitada de psicólogos e terapeutas online) também possua ferramentas de comunicação integradas dentro da ferramenta, além de manter a relação entre profissional e paciente de forma recorrente. 
