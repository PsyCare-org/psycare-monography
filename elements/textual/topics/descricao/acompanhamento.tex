\section{Tela de acompanhamento}
\label{sec:acompanhamento}

Ao estar autenticado, seja como usuário ou como profissional, a tela de acompanhamento disponibiliza todas as funcionalidades relacionadas ao acompanhamento. Além dos botões auxiliares de \quotes{Chat} para ir até o chat e \quotes{Chamada} para ir até a videochamada. A página possui cinco fluxos, que são acessados pelo menu lateral esquerdo.

\subsection{Tela de acompanhamento - Detalhes}
\label{sec:acompanhamentoDetalhes}
Utilizando o menu lateral, clicando sobre \quotes{Detalhes}, chega-se à seção de detalhes do atendimento, que por sua vez traz um resumo do mesmo, informando o horário dos encontros, a quantidade de encontros realizados, o status do atendimento, a quantidade de afazeres cadastrados e a data de início do atendimento. Os detalhes estão disponíveis tanto para o profissional quanto para o usuário. A \refImage{attendanceDetails} demonstra essa seção.

\image
    {PsyCare - Acompanhamento: Detalhes}
    {attendanceDetails}
    {data/figures/attendance-details.png}
    {width=1\textwidth}
    {Autor}

\subsection{Tela de acompanhamento - Prontuário}
\label{sec:acompanhamentoProntuario}
Ao clicar sobre \quotes{Prontuário} no menu lateral esquerdo, torna-se possível chegar até o prontuário médico do atendimento. Nele é possível criar e editar o prontuário do paciente informando a demanda inicial, a história regressa, o plano de intervenção e as evoluções do tratamento. Por tratar-se de informações extremamente sensíveis, este submenu aparece apenas para o profissional. Partindo disso, a \refImage{attendanceMedicalRecord} exibe uma demonstração desse fluxo.

\image
    {PsyCare - Acompanhamento: Prontuário}
    {attendanceMedicalRecord}
    {data/figures/attendance-medical-record.png}
    {width=1\textwidth}
    {Autor}

\subsection{Tela de acompanhamento - Afazeres}
\label{sec:acompanhamentoAfazeres}
A subseção de afazeres contém as tarefas e afazeres relacionados ao acompanhamento. Aqui, o profissional disponibiliza ao paciente tarefas e afazeres que auxiliam o processo da terapia, como recomendações de leitura, prática de exercícios físicos/mentais, lembretes, e assim por diante. A \refImage{attendanceFollowUp} demonstra esse fluxo.

\image
    {PsyCare - Acompanhamento: Afazeres}
    {attendanceFollowUp}
    {data/figures/attendance-follow-up.png}
    {width=1\textwidth}
    {Autor}

Dentro desse contexto, o profissional pode utilizar o botão \quotes{Criar afazer} no topo da seção para cadastrar novos itens, informando título, tipo (sendo este \quotes{Normal} para afazeres que devem possuir um rastreamento de completude, ou sendo \quotes{Contínuo} para afazeres a serem realizados continuamente durante o processo terapêutico) e descrição. Concomitantemente a isso, à direita de cada afazer também é possível editá-lo ou excluí-lo. A \refImage{attendanceFollowUpCRUD} exemplifica os fluxos de CRUD.

\image
    {PsyCare - Acompanhamento: Afazeres, CRUDs}
    {attendanceFollowUpCRUD}
    {data/figures/attendance-follow-up-crud.png}
    {width=1\textwidth}
    {Autor}

Para o paciente, além da listagem, a única ação disponível é a \quotes{Concluir afazer}, que está disponível apenas nos afazeres do tipo \quotes{Normal}. A \refImage{attendanceFollowUpFinish} demonstra um exemplo desse fluxo.

\image
    {PsyCare - Acompanhamento: Afazeres, concluir afazer}
    {attendanceFollowUpFinish}
    {data/figures/attendance-follow-up-finish.png}
    {width=.8\textwidth}
    {Autor}

\subsection{Tela de acompanhamento - Avaliar}
\label{sec:acompanhamentoAvaliar}

Estando disponível apenas ao usuário, a subseção de avaliação permite avaliar o atendimento. Nela, ao informar apenas o valor (de um a cinco) e a descrição, torna-se possível ao paciente avaliar anonimamente o serviço prestado pelo profissional. Uma vez avaliado, torna-se possível editar ou excluir a avaliação. A \refImage{attendanceRating} exibe essa tela.

\image
    {PsyCare - Acompanhamento: Avaliação}
    {attendanceRating}
    {data/figures/attendance-rating.png}
    {width=1\textwidth}
    {Autor}

A avaliação que aqui é gerenciada aparece juntamente a outras avaliações que o respectivo profissional recebeu na tela do profissional, citada anteriormente no subcapítulo \ref{sec:profissional}.

\subsection{Tela de acompanhamento - Encontros}
\label{sec:acompanhamentoEncontros}

Disponível apenas ao profissional, a subseção de encontros disponibiliza ao psicólogo gerenciar seu registro de encontros realizados durante o atendimento. Dentro desse contexto, o profissional realiza suas anotações de cada sessão realizada, auxiliando-o a se organizar para fornecer o atendimento de maneira mais efetiva. Cada registro de encontro consiste da data do encontro, do resumo, do relatório, da análise teórica e das observações gerais. A \refImage{attendanceMeeting} demonstra essa subseção.

\image
    {PsyCare - Acompanhamento: Encontros}
    {attendanceMeeting}
    {data/figures/attendance-meeting.png}
    {width=1\textwidth}
    {Autor}

Para criar um novo encontro, a plataforma disponibiliza o botão \quotes{Criar encontro} no topo da subseção, que por sua vez abre uma janela de diálogo onde o profissional informa os dados necessários para o cadastro. Juntamente a isso, embaixo de cada registro de encontro também há a possibilidade de editar e excluir o item. A \refImage{attendanceMeetingCRUD} exemplifica esses três fluxos.

\image
    {PsyCare - Acompanhamento: Encontros, CRUDs}
    {attendanceMeetingCRUD}
    {data/figures/attendance-meeting-crud.png}
    {width=1\textwidth}
    {Autor}

Aliado a tudo isso, a subseção também disponibiliza filtros de pesquisa dos encontros, possibilitando ao profissional procurar por registros específicos, informando uma data inicial e final e o conteúdo. Para o campo de \quotes{Conteúdo} recomenda-se utilizar palavras-chave, uma vez que o mesmo procurará qualquer ocorrência dentro do resumo, relatório, análise teórica, ou observações gerais dos encontros.

\subsection{Tela de acompanhamento - Encerrar acompanhamento}
\label{sec:acompanhamentoEncerrar}

Por fim, o fluxo de \quotes{Encerrar atendimento} disponibiliza-se tanto para o profissional quanto para o paciente para permitir com que o atendimento seja encerrado permanentemente. A tela apresenta um aviso detalhado e esclarecedor dos impactos causados pelo encerramento juntamente com o botão de confirmação da ação. A \refImage{attendanceDelete} exibe esse fluxo.

\image
    {PsyCare - Acompanhamento: Encerrar acompanhamento}
    {attendanceDelete}
    {data/figures/attendance-delete.png}
    {width=1\textwidth}
    {Autor}

Ao confirmar a ação, o atendimento passa a fazer parte do histórico de acompanhamentos da pessoa. O usuário também é direcionado automaticamente à tela de histórico do acompanhamento em questão.