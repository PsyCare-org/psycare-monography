\section{Arquitetura do sistema}
\label{sec:arquitetura}

A arquitetura base do sistema divide-se entre cinco módulos distintos. O \textSpecial{Cliente} comunica-se com o \textSpecial{Servidor}, que por sua vez realiza as devidas intercomunicações com o \textSpecial{Banco de Dados} e o \textSpecial{Serviço Externo B}, além disso, tanto \textSpecial{Cliente} quanto \textSpecial{Servidor} comunicam-se diretamente com o \textSpecial{Serviço Externo A}. A \refImage{arquitetura} esclarece essa arquitetura apresentando todo o fluxo de comunicação existente.

\image
    {Diagrama da arquitetura do sistema}
    {arquitetura}
    {data/figures/arquitetura.png}
    {width=1\textwidth}
    {Autor}

A estrutura principal do sistema encontra-se no conjunto \textSpecial{Cliente}, \textSpecial{Servidor} e \textSpecial{Banco de Dados}. Nesse contexto, o \textSpecial{Serviço Externo B} entra em ação no momento de cadastro de um novo profissional do sistema, inserindo seus dados no Cadastro Nacional de Psicologas(os), a fim de garantir que o psicólogo em questão trata-se de um profissional credenciado de acordo com a legislação brasileira. Por fim, o \textSpecial{Serviço Externo A} faz-se proficiente para o gerenciamento das videochamadas da aplicação, comunicando-se diretamente com o \textSpecial{Cliente} e \textSpecial{Servidor}.
