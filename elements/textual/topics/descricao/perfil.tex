\section{Tela de perfil}
\label{sec:perfil}

Estando autenticado, para auxiliar tanto o usuário quanto o profissional a gerenciar seus dados dentro da plataforma, o usuário pode clicar sobre o ícone do avatar no topo da página, e logo após em "Perfil" para chegar na tela em questão, conforme demonstra a \refImage{profileRedirect}. A página possui quatro fluxos, que podem serem acessados no menu lateral esquerdo.

\image
    {PsyCare - Chegando à tela de perfil}
    {profileRedirect}
    {data/figures/profile-redirect.png}
    {width=1\textwidth}
    {Autor}


\subsection{Tela de perfil - Dados}
\label{sec:perfilDados}
Ao utilizar o menu lateral esquerdo e clicar sobre "Dados", torna-se possível chegar à tela que permite tanto o usuário quanto o profissional gerênciar as informações pessoais que são utilizados dentro da plataforma.

Ao usuário, os campos nome, sobrenome, celular, gênero e data de nascimento são passíveis de alteração. Entretanto, o campo e-mail dispõe-se apenas como leitura. A \refImage{profileDataUser} demonstra o fluxo comentado.

\image
    {PsyCare - Perfil: Dados, visão do usuário}
    {profileDataUser}
    {data/figures/profile-data-user.png}
    {width=1\textwidth}
    {Autor}

Já para o profissional, além dos campos já disponíveis ao usuário, também há idiomas de atendimento, resumo (onde informa-se um breve texto que servirá para introduzir o profissional os usuários da plataforma), experiência (tópicos que o profissional possui experiência em abordar), especialidades (tópicos onde o profissional é especialista em abordar), formação (formação acadêmica do profissional) e descrição pessoal. A \refImage{profileDataProfessional} demonstra a tela.

\image
    {PsyCare - Perfil: Dados, visão do profissional}
    {profileDataProfessional}
    {data/figures/profile-data-professional.png}
    {width=.8\textwidth}
    {Autor}

\subsection{Tela de perfil - Avatar}
\label{sec:perfilAvatar}
Dentro do submenu de "Avatar", tanto usuário quanto profissional conseguem gerênciar sua respectiva foto de avatar dentro da plataforma, carregando uma nova imagem ou removendo a imagem atual. A \refImage{profileAvatar} demonstra a página em questão.

\image
    {PsyCare - Perfil: Avatar}
    {profileAvatar}
    {data/figures/profile-avatar.png}
    {width=.8\textwidth}
    {Autor}

\subsection{Tela de perfil - Alterar senha}
\label{sec:perfilAvatar}
Estando no submenu de "Alterar senha", a plataforma disponibiliza, para ambos os casos, a possibilidade de alterar a senha. Para realizar a operação, deve-se informar a senha atual e a nova senha (juntamente com sua conformação). A seguir, a \refImage{profilePassword} expõe esse fluxo.

\image
    {PsyCare - Perfil: Alterar senha}
    {profilePassword}
    {data/figures/profile-password.png}
    {width=.8\textwidth}
    {Autor}

\subsection{Tela de perfil - Excluir conta}
\label{sec:perfilAvatar}
Por fim, o fluxo de "Excluir conta" permite tanto ao profissional quanto ao usuário realizarem a exclusão permanente de conta e todos os dados relacionados. A subseção apresenta apenas um aviso esclarecedor e o botão de confirmação da ação, A \refImage{profileDelete} demontra-a.

\image
    {PsyCare - Perfil: Excluir conta}
    {profileDelete}
    {data/figures/profile-delete.png}
    {width=.8\textwidth}
    {Autor}