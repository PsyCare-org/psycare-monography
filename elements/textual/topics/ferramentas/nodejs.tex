\section{NodeJS}
\label{sec:nodejs}
NodeJS é um ambiente de execução de código JavaScript e TypeScript externo a um navegador \textit{web}. Este software inspira-se por sistemas como a máquina de eventos da linguagem Ruby e o Twist do Python para caracterizar-se pela sua arquitetura assíncrona e orientada por eventos \cite{Foundation2023}. 

Sua implementação é contrastante em relação à outras tecnologias pois não apresenta o modelo convencional de simultaneidade, onde conceitos de threads do sistema operacional são empregados.  Seu \textit{runtime} é executado por apenas uma \textit{thread} que executa o \textit{looping} de eventos, que perdura desde a criação da \textit{thread} até quando não há mais retornos de chamadas a serem concluídos \cite{Foundation2023}.

Dentro desse mesmo contexto, chamadas que normalmente seriam bloqueantes (que requerem recursos do sistema operacional) são realizadas assíncronamente utilizando a biblioteca libuv \cite{ClaudioWunder}. Na \refImage{nodejs} é possível visualizar um diagrama simplificado desse fluxo.

\image
    {\textit{Looping} de eventos NodeJS}
    {nodejs}
    {data/figures/nodejs.png}
    {width=.8\textwidth}
    {\citeonline{Santos2020}}

\subsection{NPM}
\label{sec:npm}
Juntamente de suas funcionalidades base, NodeJS também dispõe do NPM, um gerenciador de pacotes reutilizáveis de código aberto, permitindo maior agilidade, flexibilidade e produtividade no processo de desenvolvimento \cite{npm2022}.

Nesse contexto, o pacote \textit{Video SDK} destaca-se por ser um kit de desenvolvimento de software, que através de suas funcionalidades e APIs permite a criação de chamadas de áudio e/ou vídeo, bate-papo, gravação em nuvem, transmissão simultânea (RTMP) e transmissão interativa ao vivo (HSL) \cite{Video2023}. Internamente, o pacote emprega e abstrai a utilização da tecnologia \textit{WebRTC}, que por sua vez trata-se de um protocolo de comunicação de código aberto que permite comunicação em tempo real e troca de dados entre diferentes navegadores e dispositivos, permitindo assim, a transmissão de som, vídeo e dados via internet via P2P \cite{Network2023}. Seu funcionamento se dá através do modelo \textit{freemium}, onde é possível usufruir de suas funcionalidades básicas até determinado ponto de maneira gratuita, e após isso é necessário realizar pagamento para continuar utilizando-o. 

Além disso, \textit{Socket.IO} também sobressai-se por ser uma biblioteca de código aberto que permite uma comunicação \textit{client-server} de baixa latência, bidirecional e baseada em eventos. Sua utilização permite a implementação dos protocolos \textit{WebSocket}, \textit{WebTransport} e \textit{HTTP long-polling} para produzir comunicações e atualizações em tempo real dentro da \textit{web}. 