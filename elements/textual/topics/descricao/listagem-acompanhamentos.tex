\section{Tela de listagem de acompanhamentos}
\label{sec:acompanhamentos}

Estando autenticado, e sendo tanto um usuário quanto um profissional, ao clicar sobre "Acompanhamentos" no cabeçalho da aplicação, chega-se à tela de acompanhamentos. Nessa tela, a aplicação disponibiliza os acompanhamentos ativos da pessoa em questão, listando-os de maneira sucinta, apresentando apenas as informações indispensáveis de cada acompanhamento. Paralelamente a isso, também disponibiliza-se a agenda diária da pessoa, mostrando por horário os encontros que a mesma possui naquele dia. A \refImage{attendances} demonstra tal visão da tela.

\image
    {PsyCare - Listagem de acompanhamentos}
    {attendances}
    {data/figures/attendances.png}
    {width=1\textwidth}
    {Autor}

Nesta tela, ao clicar em um dos eventos da agenda diária, a plataforma abre uma janela de diálogo, permitindo ir rapidamente à tela de videochamada, à tela de chat ou à tela do acompanhamento do evento em questão. A \refImage{attendancesModal} exibe tal janela de diálogo.

\image
    {PsyCare - Listagem de acompanhamentos, evento da agenda}
    {attendancesModal}
    {data/figures/attendances-modal.png}
    {width=.7\textwidth}
    {Autor}

Além dos recursos já citados, no topo da tela, caso a pessoa possua alguma solicitação de atendimento, o botão "Solicitações de Atendimento" passa a ser apresentado. Ao clicar, sendo um profissional, lista-se todas as solicitações que o mesmo possui, onde é possível enviar mensagens ao paciente e aceitar ou recusar o atendimento. E ao clicar sendo um usuário, também lista-se todas as solicitações que o mesmo possui, mas desta vez, é possível apenas enviar mensagens ao profissional e visualizar o perfil do mesmo. A \refImage{attendancesPending} demonstra ambos os casos.

\image
    {PsyCare - Listagem de acompanhamentos, solicitações}
    {attendancesPending}
    {data/figures/attendances-pending.png}
    {width=.8\textwidth}
    {Autor}

Por fim, ao clicar sobre um dos acompanhamentos listados em tela, chega-se à página do acompanhamento em questão.