\section{Tendência de integração tecnológica}
\label{sec:tendenciaDeIntegracaoTecnologica}

De acordo com o resumo científico \textit{Mental Health and COVID-19: Early evidence of the pandemic’s impact} publicado pela \citeonline{OMS2022}, a disrupção causada principalmente pela pandemia SARS-CoV-2 levou a profissionais da área de saúde mental relatarem cortes abruptos por inúmeros motivos em acompanhamentos psicológicos de seus pacientes.

Ainda de acordo com o resumo, para mitigar os danos causados pelos cortes, grande parte dos profissionais da área alegaram que passaram a integrar em seus serviços tecnologias digitais, com consultas, terapias e acompanhamentos pelo telefone ou por meio de plataformas de videoconferência e aplicações \textit{web}.

Essa migração provou-se ser mais flexível e assertiva para grupos específicos de pessoas, em especial pessoas mais novas e/ou financeiramente independentes com seu próprio espaço privado. Além disso, diversas revisões coletadas pelo resumo também relatam avaliações positivas dessa mudança em termos de custo-efetividade, aceitabilidade e conveniência, especialmente para transtornos mentais comuns e para atendimento ambulatorial.
