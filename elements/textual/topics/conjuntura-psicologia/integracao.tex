\section{Tendência de integração tecnológica}
\label{sec:tendenciaDeIntegracaoTecnologica}

Em primeiro plano, durante o período de levantamento de requisitos deste trabalho, onde foi conversado de maneira informal com inúmeros profissionais e pacientes da área, foi compreendido que há certa informalidade na procura de profissionais da área, onde na maioria das vezes é realizada através de pesquisas rápidas em ferramentas de busca \textit{online} e/ou comunicação boca a boca por familiares, amigos ou conhecidos. O principal sintoma causador relatado pelos profissionais encontra-se no código de ética do profissional psicólogo \cite{Psicologia2005}, que ao seguir a tendência de outras áreas médicas acaba por limitar severamente às formas de propaganda do profissional, que dentre outras delimitações, impede por exemplo a utilização de preço como forma de propaganda. Restando assim ao profissional recorrer a ferramentas de busca e/ou localização \textit{online}.

Partindo para o contexto global, de acordo com o resumo científico \textit{Mental Health and COVID-19: Early evidence of the pandemic’s impact} publicado pela \citeonline{OMS2022}, a disrupção causada principalmente pela pandemia SARS-CoV-2 levou a profissionais da área de saúde mental relatarem cortes abruptos por inúmeros motivos em acompanhamentos psicológicos de seus pacientes.

Ainda de acordo com o resumo, para mitigar os danos causados pelos cortes, grande parte dos profissionais da área alegaram que passaram a integrar em seus serviços tecnologias digitais, com consultas, terapias e acompanhamentos pelo telefone ou por meio de plataformas de videoconferência e aplicações \textit{web}. Essa migração provou-se ser mais flexível e assertiva para grupos específicos de pessoas, em especial pessoas mais novas e/ou financeiramente independentes com seu próprio espaço privado. Além disso, diversas revisões coletadas pelo resumo também relatam avaliações positivas dessa mudança em termos de custo-efetividade, aceitabilidade e conveniência, especialmente para transtornos mentais comuns e para atendimento ambulatorial.

Por fim, aliado a estas mudanças, inúmeros estudos ponderam e comprovam que a eficácia do emprego de terapia online é muito similar ao que é alcançado pelos métodos tradicionais, sendo este o mais recomendado quando a abordagem presencial não se faz presente por quaisquer motivos. Tais conclusões são evidênciadas por estudos sobre tratamentos de pacientes com transtornos de ansiedade generalizada \cite{Eilert2020}, transtornos de estresse pós-traumático \cite{Simblett2017}, transtornos alimentares \cite{Loucas2014}, e pesquisas sobre a efetividade geral de atendimentos \textit{online} \cite{Barak2008}.
